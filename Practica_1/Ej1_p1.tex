\documentclass{article}
\usepackage[utf8]{inputenc}

\documentclass[fleqn, 10pt]{article}


% Paquetes necesarios
\usepackage[utf8]{inputenc}
\usepackage{amsthm, amsmath}
\usepackage{nccmath} %Para centrar ecuaciones
\usepackage{graphicx}
\usepackage{enumitem}

% Personalizo mi alfabeto
\DeclareMathAlphabet{\pazocal}{OMS}{zplm}{m}{n}
\newcommand{\Lb}{\pazocal{L}}

% Definimos los entornos para definiciones, teoremas, etc...
\theoremstyle{plain}
\newtheorem{proposicion}{Proposición}

\theoremstyle{definition}
\newtheorem{definition}{Definición}[section]
\newtheorem{example}{Ejemplo}[section]

%Definimos el título
\title{Teoría de Autómatas y Lenguajes Formales\\[.4\baselineskip]Práctica 1: Latex y expresiones regurales}
\author{Santiago, Díez Cantisano}
\date{\today}

%Comienzo del documento
\begin{document}

%Generamos el título
\maketitle

\section{Potencia de una relación $R^n$}

\begin {definition}
Given $ R \subseteq A \times A $,

\begin{center}
$ R^n = \begin{cases}
R & n=1 \\
\left\{(a,b) \suchthat \exists x \in A, (a,x)\in R^{n-1}\wedge (x,b)\in R\right\} & n > 1

\end{cases}$
\end{center}

\end {definition}

\begin{example}
Given the binary relation $R = \{(1, 1), (1, 2), (2, 3),(3,4)\}$


\qquad $R^3 = \{(1, 1), (1, 2), (1, 3),(1,4)\}$


\end{example}


\end{document}